\documentclass[12pt]{article}
\usepackage{amsmath}
\usepackage{graphicx}
\usepackage{defs}
\usepackage{natbib}
 \newcommand{\edotv}{\dot{\epsilon}^{v}}
 \newcommand{\edotb}{\edotv_{b}}
 \newcommand{\edotsz}{\edotv_{sz}}
 \newcommand{\Di}{\mathrm{Di}}

\begin{document}
\section{A simplified dynamical system model for ``viscous Earthquakes''}
\label{sec:simpl-dynam-syst}

Here we set up the equations governing a simplified 1-D visco-elastic system with a single non-linear flow law for viscous relaxation.  The primary variables are the stress $\tau$ in the system and the Temperature $T$ in the shear-zone.  The shearzone is assumed to be a narrow, small-grainsized region of width 2$h$ in a larger system of size 2$L$.  The dimensionless dynamical system can be written
\begin{align}
  \label{eq:1}
  \diff{\tau}{t} =& 1  - 2
                  \left[
                  (1-l)\edotb + l\edotsz
                  \right]\\
 \diff{T}{t} =& \frac{1}{\Pe}(1-T) + \Di\tau
                 \left[
                 (1-l)\edotb + l\edotsz(T)
                 \right]\\
\end{align}
where
\begin{displaymath}
  \edotv(T,\tau,d) = A\tau^{n} d^{m}\exp
  \left[
    \frac{-\Tstar}{T}
  \right]
\end{displaymath}
is the generic viscous flow law and $\edotb=\edotv(T_{b},d_{b},\tau)$ is the strain-rate in the background material with grain-size $d_{b}$ and temperature $T_{b}$ and $\edotsz=\edotv(T,d_{sz},\tau)$ which is the srainrate in the shearzone.  To close the problem requires some statement of the temperature in the background $T_{b}$ and any initial grain-size variation $d(z)$.

Our first model assumes $T_{b}=1$ (stays at the initial temperature) and $d_{b}=1$ is the reference grain-size.
In the shearzone $T$ can vary but $d_{sz}<<1$.

Initial conditions are that $\tau(0)=0.$, $T(0)=1.$

Values for rheology parameter will be taken from \cite{kelemen_periodic_2007}
\bibliography{Tfshearzones}
\bibliographystyle{plainnat}

\end{document}
